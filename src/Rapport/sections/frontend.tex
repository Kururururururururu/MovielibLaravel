\section{Frontend}


We have developed the front-end of the application using HTML, CSS and JavaScript.
However, Blade templates have also been used since we are working with Laravel which is a framework that incorporates them.
Blade allows you to use PHP within your HTML, which can be very handy in order to display data from APIs, databases or other sources.\newline

\textbf{HTML 5:} \newline
HTML 5 is the most recent version of HTML, and it also introduced some new semantic tags.
Semantic tags on the other hand can be very good for accessibility and SEO.
Also, they make it easier for developers to understand page structure as well as the purpose of each section.
For example, \textless{}header\textgreater{}, \textless{}nav\textgreater{}, \textless{}main\textgreater{}, and \textless{}footer\textgreater{} among others have been used to define our pages’ header, navigation bar and footer.\newline

In our project, we tried to utilize these semantic tags as best as possible in order to identify different sections on the page.
These include movie information, comments and casts when viewing a specific movie on its own page.
This makes it easy to understand each section’s purpose by giving an overview of how each section has been divided into.
Using more descriptive markup such as the \textless{}section\textgreater{} instead of generic \textless{}div\textgreater{} tags provides more meaningful semantics.\newline

\textbf{CSS:} \newline
CSS is used for setting up the style and visuals of the HTML page elements,
and these define features such as the layout, colors, fonts, as well as the overall page structure.
It improves the visual appearance and can also make it user-friendly with things like animations on hover, transitions or a number of other CSS3 capabilities that may indicate to a user that an element can be clicked.\newline

In our application we didn’t want to use the default styles that the browser have, so we used CSS to style the page and application to what we wanted.
However, CSS can become difficult to navigate through and maintain.
The reason behind this is, you never know if a style has been used somewhere else in the application.
This can cause inadvertent changes in other sections of the app which are often hard to debug.
For this reason especially being a group project, which will most of the time make it even more difficult, we had different CSS files for different pages unless they were common components across pages.
So, as far as that problem was concerned we also made sure that descriptive class names were used only along with targeting IDs and classes that were specific to the page. This way others would remain unchanged by mistake.
There is a downside to doing this, however, as it can lead to a lot of duplicate code, which can also make it hard to maintain, as it is likely that at some point, there will be unused styles in those CSS files.
We weighed the pros and cons of this approach and decided that it was the best approach for our project.\newline

\textbf{JavaScript:} \newline
JavaScript is a programming language that helps in making web pages interactive, dynamic, and add or remove elements from the page, change the styles of elements as well as make API calls.
Since Laravel is a server-side application most of our data are fetched from the server on page load.
But we also have some dynamic elements on the page such as the comment section on a movie, which can be fetched from server when user clicks comment button by making an API call to server, the watchlist button for correctly displaying current state of movie’s existence in user's watchlist and, rating system used to rate movie from 1 to 5 stars.\newline

The way that you incorporate JavaScript into your HTML is by using the \textless{}script\textgreater{} tag. It is not only used for including JavaScript files but can also be used for writing JavaScript code directly in the HTML file.
We have gone with the former approach, because it is less complex to maintain and debug since you can easily see which JavaScript files are included in the page.
Nevertheless, we have also utilized this last option but only for minor scripts that are unique to that specific page.\newline