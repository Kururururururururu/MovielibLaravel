\documentclass[letterpaper,twocolumn]{article}

\usepackage{listings}
\usepackage{pdflscape}
\usepackage{booktabs}

\newcommand{\myparagraph}[1]{\vspace{0.1cm}\noindent \textbf{\textit{#1.}}}

\title{Individual Extension Report\\Web Technologies (E23): T510048101-1-E23}
\author{Daniel Bermann Schmidt\\University of Southern Denmark}
\date{\today}
\begin{document}


\maketitle

\section{Introduction}

\section{Reflections}

\section{Individual Extension}

In my invidual extension I have chosen to implement React into the project.
React is a JavaScript library for building user interfaces.

\subsection{Why React?}

Building web UIs, it can be a challenge to keep the code organized as the project grows.
React solves this problem by allowing you to split the UI into independent, reusable pieces.
This can make the code more organized.

\subsection{How and what did I implement with React?}

I started implementing React by installing the npm packages `react' and `react-dom' into the project.
Then I included the JSX page file into the Blade template, and a div with a specific ID, where the React component would be rendered.

Our project was built on Laravel which is a server-side rendered application.
A problem was how I could get data from the server into the React component without calling the API from the client-side.
I solved this in the Blade template with `json\_encode()' in the data attribute of the HTML div for the React component, then parsing it again in the component.

\subsection{Reflection on the implementation}

I split the UI into components and made them reusable.
This will make it easier to maintain.
And by injecting the data into the data attribute,
the React components will still display the correct data without calling the API on page load.

\section{Security Reflections}

In terms of security, we have login and register functionality.
We don't store any passwords in plain text, but we hash them with bcrypt.


\section{Performance \& Scalability Reflections}

\section{Conclusions}


\end{document}

