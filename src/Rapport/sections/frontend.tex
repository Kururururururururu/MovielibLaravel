\section{Frontend}

In our Laravel application, we have used HTML, CSS, and JavaScript to build the front-end side of the application.
Instead of directly writing HTML, we have utilized Blade templates, which is the built-in templating engine in Laravel.
Blade allows us to incorporate PHP code into our HTML, which is particularly useful for displaying data from APIs, databases, or other sources.\newline

\textbf{HTML 5:} \newline
HTML 5 is the latest version of HTML, introducing additional semantic tags that enable us to create content with proper structure and meaning.
By using semantic tags, we ensure that our web pages are accessible, readable for search engines, and compatible with screen readers.
Moreover, semantic tags make it easier for developers to understand the page structure and the purpose of each section.
For instance, we have utilized tags like \textless{}header\textgreater{}, \textless{}nav\textgreater{}, \textless{}main\textgreater{}, and \textless{}footer\textgreater{} to define the header, navigation bar, and footer of our pages.\newline

In our project, we have, to the best of our ability, tried to make use of these semantic tags to define the different sections of the page.
For example, on the page for a specific movie, we have separate sections for movie information, comments, and casts.
This division into sections gives a clarity and makes it easier to understand the purpose of each section.
Using the \textless{}section\textgreater{} tag instead of generic \textless{}div\textgreater{} tags provides more descriptive and meaningful markup.\newline

\textbf{CSS:} \newline
CSS is used to define the style and visuals of the HTML elements on the page, defining aspects such as layout, colors, fonts, and overall page structure.
Its purpose is to enhance the visuals and can also, if used correctly, enhance user-friendliness of the page with animations, transitions and things like hover effects, which can be used to indicate that an element is clickable and interactive.\newline

In our application we didn't want the default styles, so we used CSS to style the page and application to what we wanted.
However, CSS can become challenging to navigate and extend. This is because you never know if a style is used somewhere else in the application.
This can lead to unintentional changes in other parts of the application, which can be hard to debug.\newline

To avoid this, especially since this was a group project, which can make the problem even more apparent, we created a separate CSS file for each page unless it was a component that was used on multiple pages.
In order to combat this problem, we also tried to use descriptive class names together with only targeting IDs and classes that were specific to the page.
This way, we could avoid unintentional changes in other parts of the application.
There is a downside to doing this, however, as it can lead to a lot of duplicate code, which can also make it hard to maintain, as it is likely that at some point, there will be unused styles in those CSS files.\newline

\textbf{JavaScript:} \newline
JavaScript is a programming language that is used to make web pages interactive and dynamic like adding and removing elements from the page, changing the style of elements, and making API calls.
Since Laravel is a server-side application, most of our data is fetched from the server when the page is loaded.
But we also have some dynamic elements on the page, such as the comment section on a movie,
which is fetched from the server when the user clicks on the comment button by making an API call to the server,
the watchlist button to correctly display the current state of the movie's existence in the user's watchlist,
and the rating system, which is used to rate a movie from 1 to 5 stars.\newline

The way that you incorporate JavaScript into HTML is by using the \textless{}script\textgreater{} tag.
This tag can be used to include JavaScript files, but it can also be used to write JavaScript directly in the HTML file.
We have use the first approach, as it is easier to maintain and debug, as you can easily see which JavaScript files are included in the page.
However, we have also used the latter but only for small scripts that are only used on that page.\newline







The first task of this project is to build the front-end side of the application.
In this section, we will provide a technical description of the implementation details related to HTML 5, CSS, and JavaScript.\newline

\begin{itemize}
    \item HTML 5: We will explain which HTML tags we have used, where we have used them, and the reasons behind our choices.
    \item CSS: We will discuss why and how we have utilized CSS, including interesting selectors and declarations, and how it is integrated into the application.
    \item JavaScript: We will explain why and how we have used JavaScript in our application, including any noteworthy behaviors and how they are incorporated into the application.
\end{itemize}

\myparagraph{Resources} Lectures 1 to 3.

\myparagraph{Length} 2 columns.
