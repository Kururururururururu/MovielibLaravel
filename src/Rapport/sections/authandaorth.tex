\section{Authentication and Authorization}

To help fortify the systems overall security, it is very
important that the system knows which users has
access to certain things and can identify users that does not have
access to the given resource. This is done by implementing authorization
and authentication on our system.

\textbf{Authentication:} 

Authetication the process of verifying the
identity of a user. In our system we have two distinct
user categories "Guest" and "User". The "Guest" user
is the default user and can only access the home
page and the login page. The "User" user is the
authenticated user that can access all the pages in
the system.
The implementation of the authentication is done by
verifying the user's identity. The system asks for a
username and password when you login. The system
then checks if the username and password match the
ones stored in the database. If they match, the user
is authenticated and can access the system. If the
username and password do not match anything in
the systems database, the user is not authenticated
and can only access a limited set of resources on the
page. \newline

\textbf{Authorization:} 

When the user is authenticated,
the system needs to check if the user is authorized
to access the resource, he is trying to access. In
our system we have two distinct user categories
"Guest" and "User". The "Guest" user is the default
user and can only access the home page and the
login page. The "User" user is the authenticated
user and can access all the pages in the system.
The implementation of the authorization is done by
checking the user's role. The system checks if the
user is a "Guest" or a "User". This is done within
the system to ensure that the user is authorized to
access the resource he is trying to access. If the
guest is trying to access something that he should
not be able to access, the system will redirect him to
the login page. This will help fortify the security of
the system and ensure that only the right users can
access the right resources. As a result, the system
will be more secure.

% [kommentar 1] Skal dette med linje 51-63? Det er 99% gentagelse af det ovenfor.
To give a better indication of the different user roles we have in our system, we have:
\begin{itemize}
    \item \textbf{Guest:} The default user, can only access the home page and the login page.
    \item \textbf{User:} The authenticated user, can access all the pages in the system.
\end{itemize}

% [kommentar 2] giver eller implementere vi different user roles
By giving our website different user roles, we can
ensure that the system, is safe against incoming
attacks. It also makes it so, that people can not access
information in the system that they should not be
able to see without having created a user.
It also works the other way around. Only people
that are created on the system, can comment on the
different movies on the website, while the unauthorized users can only see the different comments. If
we were to develop the system further we could add
some more user roles with different authorizations.
We could for example make an admin user, which has access
to everything including the profiles of the different
users. With the way we have implemented the
authorization and authentication in our system, we
can easily scale the different user's roles, without any
complications. \newline

