\section{introduction}

Our group has developed a movie library that acts as an independent service in which users can get an overview of a broad range of existing movies. The movie library enables users to create a watchlist and keep an overview of the added movies. Further, the web service features a rating system allowing users to rate movies with 1-5 stars. Additionally, a public comment section is available. Thus, allowing users to make comments, which can be read by others.\newline
The movie library differentiates it's watchlist system from existing streaming services by providing access to a wider collection of movies. Thereby, creating an online environment in which users can create a collected overview.

characters with spaces: 688
words: 112

------ guide: ------

\noindent The goal of the introduction is to let the readers (the professor and TAs) know the topic of your work and the main takeaways of it.
The introduction should be broad enough to understand the document without reading it and specific enough to let the reader know: \textit{If you are interested in this topic, you should read this work}.

\noindent In the context of the Web Technologies course, the introduction should clearly describe:

\begin{itemize}
    \item Motivation: what problem does this work try to solve (and why is it important)
    \item Project: clearly describes the topic the group chose to work with
    \item Contributions: main takeaways that readers will get from reading this work
\end{itemize}


\noindent Remember to keep this and all other sections within the page and column limits.
It is your responsibility to describe first the most important and interesting aspect of each section.
That way, you can leave behind non-interesting and repeated information more easily.

\myparagraph{Length} Half a column.
